\documentclass{beamer}%
\usepackage[T1]{fontenc}%
\usepackage[utf8]{inputenc}%
\usepackage{lmodern}%
\usepackage{textcomp}%
\usepackage{lastpage}%
\usepackage{geometry}%
\geometry{tmargin=1cm,lmargin=1cm,margin=1cm}%
%
\title{Implementación del algoritmo de Quine{-}McCluskey}%
\author{Sebastian Hidalgo, Daniela Quesada, Joel Chavarria}%
\date{\today}%
%
\begin{document}%
\normalsize%
\section{Titlepage}%
\label{sec:Titlepage}%
\maketitle

%
\newpage%
\section{Quine{-}McCluskey}%
\label{sec:Quine{-}McCluskey}%
Quine{-}McCluskey%
\newline%
\newline%
El procedimiento de Quine{-}McClusky parte del hecho de que una ecuación booleana está descrita por sus mintérminos.%
\newline%
\newline%
Para el ejemplo a elaborar se utilizan los siguientes minterminos: %
{[}2, 3, 5, 10{]}

%
\newpage%
\section{Primer Paso}%
\label{sec:PrimerPaso}%
Paso 1: %
\newline%
Para el primer paso se pasan los minterminos a su representacion Binaria y se procese a agrupar en una tabla por su cantidad de 1's%
\newline%
\newline%
\newline%
%
\subsection{Tabla agrupada de minterminos}%
\label{subsec:Tablaagrupadademinterminos}%
\begin{tabular}{| c | c | c |}%
\hline%
Grupo&Mintermino&Binario\\%
\hline%
1&&\\%
&2&0010\\%
\hline%
2&&\\%
&3&0011\\%
&5&0101\\%
&10&1010\\%
\hline%
\hline%
\end{tabular}

%
\newpage%
\section{Segundo paso}%
\label{sec:Segundopaso}%
Paso 2:%
\newline%
Para el siguiente paso se analizan los implicantes primos de cada grupo, asi mismo tambien se crean las tablas de estos mismos con su representacion y se saca cada implicante por grupo.\newline%
\newline%
%
De la tabla anterior se pueden obtener Elementos que son Implicantes Primos: 0101%
\newline%
\newline%
%
\subsection{Tabla agrupada de minterminos}%
\label{subsec:Tablaagrupadademinterminos}%
\begin{tabular}{| c | c | c |}%
\hline%
Grupo&Mintermino&Binario\\%
\hline%
0&&\\%
&2, 3&001{-}\\%
&2, 10&{-}010\\%
\hline%
\hline%
\end{tabular}%
\newline%
\newline%

%
De la tabla anterior se pueden obtener Elementos que son Implicantes Primos: {-}010, 001{-}

%
\newpage%
\section{Tercer paso}%
\label{sec:Tercerpaso}%
Paso 3:%
\newline%
Para el siguiente paso se muestran todos los implicantes primos que se han encontrado, luego se crea una tabla donde se representan estos minterminos esenciales y se muestran para que mintermino es esencial.\newline%
\newline%
%
Todos los Implicantes Primos: {-}010, 001{-}, 0101%
\newline%
\newline%
%
\subsection{Tabla procesar minterminos}%
\label{subsec:Tablaprocesarminterminos}%
\begin{tabular}{| c |c |c |c |c |}%
\hline%
Mintermino&2&3&5&10\\%
\hline%
2,10&X& & &X\\%
\hline%
2,3&X&X& & \\%
\hline%
5& & &X& \\%
\hline%
\end{tabular}%
\newline%
\newline%

%
\newpage%
\section{Cuarto paso}%
\label{sec:Cuartopaso}%
Paso 4:%
\newline%
Para el siguiente paso se toman todos los implicantes luego de analizarlos anteriormente y simplificar respectiamente las posiciones de la X. \newline%
Luego se muestra la funcion correspondiente simplificada. \newline%
\newline%
%
Implicantes Primos Escenciales: {-}010, 001{-}, 0101%
\newline%
%
Solución: F = B'CD' + A'B'C + A'BC'D

%
\end{document}